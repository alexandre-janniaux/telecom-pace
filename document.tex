\documentclass[letterpaper, 12pt]{article} % for letter size paper
% 215.9mm × 279.4mm

\usepackage[T1]{fontenc}
\usepackage[utf8]{inputenc}
\usepackage[french]{babel}

\usepackage{tikz, background, titling, kantlipsum, setspace}
\usepackage[left=1.5in,right=1.25in,top=1.025in,bottom=.125in]{geometry}

\usetikzlibrary{calc}

\backgroundsetup{%
 position=current page.center,
 angle=0,
 scale=1,
 contents={%
  \begin{tikzpicture}%
    [
      normal lines/.style={gray, very thin},
      every node/.append style={black, align=center, opacity=1}
    ]
    \draw [line width=0.3pt,color=gray,step=0.5cm] (current page.south west) grid (current page.north east);
    %\foreach \y in {0.71,1.41,...,25.56}
    %  \draw[normal lines] (0,\y) -- (8.5in,\y);
    %\draw[normal lines] (1.25in,0) -- (1.25in,11in);
    %\node (t) [font=\LARGE, anchor=south] at ($(0,25.56)!1/2!(8.5in,25.56)$) {\thetitle};
    %\node (d) [font=\large, anchor=south west, xshift=1.5em] at (0,25.6) {\today};
    %\node (p) [font=\large, anchor=south east, xshift=-1.5em] at (8.5in,25.56) {p.~\thepage};
  \end{tikzpicture}%
}}

\renewcommand{\rmdefault}{augie}

\title{My doc}
\author{Me}

\begin{document}
  %\pagestyle{empty}
  \doublespacing{}
  %\kant[1-6]{}

{\setstretch{1.965}
Longue attente devant la porte de la salle de cours. 
Je discute non sans anxiété avec d'autres élèves de ma classe préparatoire --- taupins dans notre jargon --- qui vont bientôt subir le même sort que moi.
Quelle sensation étrange d'être anxieux spour la présentation d'un sujet que l'on connait probablement mieux que les autres?

Ça y est, la porte s'ouvre, je rentre. Un petit regard vers l'autre table, où l'on parle beaucoup trop physique pour mon esprit d'informaticien, et je m'asseois en face de mon professeur de mathématiques.

-- ``Bonjour Alexandre, comment allez vous?''
% TODO: image M.Pauly

Si l'on vous dit un jour, comme pour moi, que les professeurs du Lycée du Parc sont arrogants, méprisants, et ne font qu'essayer de vous écrasez, vous rierez de voir cette descritpion tomber devant ce professeur.


% FIXME: Typo Dialogue
-- ``Venons-en à ce pourquoi vous êtes ici, racontez-moi ce que vous faîtes''

-- ``Oui! J'ai choisi un sujet d'informatique sur l'optimisation de la planification aérienne. On part d'un certain nombre d'avions, d'aéroports et de clients, et il faut trouver les meilleures configurations des vols en fonction de plusieurs critère.''

-- ``C'est ambitieux comme sujet, vous savez comment vous y prendre?'' % FIXME : lyon "y"

-- ``En réalité, j'ai surtout choisi ce sujet comme prétexte à l'utilisation d'algorithmes génétiques.''

-- ``Et vous avez déjà codé ce genre d'algorithme?.''

-- ``Oui, évidemment.''

Pas besoin de dire que c'est le stress qui a répondu à ma place, même si j'aurai été capable d'expliquer comment cela fonctionne. 

-- ``Dans l'état actuel, une seule chose m'inquiète. 
		J'ai l'impression que je ne fais rien de théorique, et qu'expliquer le fonction de ces algorithmes sera non seulement pauvre, mais me prendra aussi tout mon temps de parole.''

-- ``Qu'est-ce que vous aimeriez ajouter?''

-- ``J'avais l'idée de mieux formaliser l'algorithme dans ce cas là et montrer que l'on converge probablement vers un optimum.''

-- ``Vous savez, il vous reste beaucoup de temps, et vous n'avez pas encore énormément de reul sur votre projet.
		Attendez un peu: à force de vivre avec le problème, vous serez à même de l'expliquer clairement et rapidement.
		Essayez de simplifier ce problème d'abord.''

-- ``D'accord, ça me parait rapide à faire.''


Les algorithmes génétiques sont des programmes extraordinaires qui font partie de la classe des méta-heuristiques -- alias algorithme tout terrain pour les intimes. 
Ils sont capables de résoudre un problème sans vraiment en connaître les caractéristiques, et fonctionnent grâce aux dures loi de l'évolution mises en lumière par Darwin -- en particulier la sélection naturelle. 
L'algorithme semble aussi fou qu'il est efficace.
Son fonctionnement peut être très simple à décrire: 
Imaginons que l'on tape plusieurs textes totalement aléatoire, en jetant des billes sur un clavier par exemple.
Prenons alors tous ces textes pour les faire lire à une communauté de lecteurs passionnés, qui va objectivement leur donner une note.
Reprenons ensuite les textes un par un, lançons une pièce, et s'il s'agit d'un pile, modifions aléatoirement une lettre dans le texte.
Sélectionnons alors les meilleurs textes, découpons une partie dans chacun d'entre eux, que l'on échange avec un autre texte considéré meilleur que les autres.
Puis on recommence le processus: c'est un vrai bazar. 
Et pourtant, c'est exactement ce que font les algorithmes génétiques.
Et évidemment, c'est diaboliquement efficace, sous réserve que la communauté de lecteur sache correctement noter les textes -- et dispose de suffisamment de café pour supporter la lecture.
%TODO: exemple robot qui se combattent ? si il y a de la place

%TODO: Separator line
Il est déjà 19 heures, j'ai réuni toutes les sources dont j'ai besoin pour mon projet. 
C'est étrange de ressortir ce Linux Pratique Essentiel, le magazine qui m'a poussé vers cette idée de sujet. 
En soi, c'est assez difficile de choisir un sujet sur lequel travailler pendant plus d'un an, avant même de le connaître,
et j'étais assez content d'être tombé dessus.
Lorsque j'avais choisi mon sujet, en fin de première année de prépa, j'avais été aidé par un ami, maintenant en thèse d'apprentissage et robotique.
Après un long brainstorming sur comment développer le sujet imposé cette année, j'avais fait ressortir deux projets assez imposants.
Le premier, c'était bien sûr celui sur la planification aérienne. 
Quant au second, il s'agissait encore d'un problème d'otpimisation, mais qui demandait de développer une intelligence à base de réseau de neurones pour optimiser la consommation énergétique des datacenters, à l'instar de ce que fait Google.

J'ai acheté des livres également, \textit{A Field Guide to Genetic Programming} qui risique de m'accompagner longtemps.
Mais je m'étais également intéressé au livre de Cédric Villani, \textit{Théorème Vivant}, pour renouer avec mon envie de faire des maths et enfin lire son travail de vulgarisation et ses talents d'écrivains. 
Mmmmh, l'heure tourne, je devrais peut être dormir, cette semaine encore va être longue à vivre\dots

%TODO Separator line
Toujours aucune issue en vue!
Je cherche désespérément comment améliorer mon sujet et le rendre intéressant, je n'arriverais pas à le présenter autrement!
Pourtant je ne l'ai même pas encore commencé, l'enthousiasme que j'avais en choisissant ce projet s'est très vite envolé avec le début de cette année.
Enfin bon, pas le temps de culpabiliser, là tout de suite, c'est pause détente. 

La cafétéria du lycée est toujours autant peuplée, mais le flux s'estompe progressivement lorsque les lycéens ont trouvé de quoi manger et qu'ils retournent travailler. 
De mon côté, je continue de lire \textit{A Field Guide to Genetic Programming}, cette passion-là ne s'est pas arrêté au moins.
% FIXME : le -- est pas génial pour écrire
Un de mon trinôme -- ce groupe composé de trois amis soudés pour affronter les examens au tableau deux fois par semaine -- commence à s'intéresser un peu à ce que je fais.

--- ``C'est pour ton TIPE, non?''

--- ``Oui\dots en quelque sorte! il faut que j'utilise ces algorithmes pour un problème d'optimisation.''

--- ``J'en ai déjà entendu parler, mais je n'ai jamais compris comment ça fonctionnait, finalement.''

--- ``C'est assez simple, tu pars d'une population de solution initiale arbitraire, et tu la fais évoluer en fusionnant les caractéristiques chez les 			bons individus et en gardant les meilleurs, en gros.''

--- ``Ah oui\dots et tu peux t'en servir pour n'importe quoi, du moment que tu peux représenter les individus?''

--- ``Pas seulement, il faut aussi avoir une bonne fonction pour leur attribuer des notes, qui soit suffisamment expressive sans demander trop de temps. 		Mais ça a été utilisé pour faire des antennes pour la {NASA}. Attends je dois avoir une photo quelque part\dots là, voilà!''

--- ``C'est assez atypique comme antenne, en effet!''

--- ``Là aussi, il y a un autre exemple avec la génération de circuit électrique pour obtenir une certaine fonctionnalité, et ça doit être possible de 			forcer l'algorithme pour qu'il réalise des circuits de taille minimale.''

Me voilà reparti dans ma lecture.

%TODO: texte avec l'image de l'antenne et une explication du travail de Holand et ...

%TODO: ligne horizontal






}
\end{document}