\documentclass{article}

\usepackage[T1]{fontenc}
\usepackage[utf8]{inputenc}
\usepackage[french]{babel}

\title{\emph{Quand la logique s'en mêle} -- résumé}
\author{Alexandre JANNIAUX}
\date{}

\begin{document}

\maketitle
\vspace{0.5cm}\hrule\vspace{0.5cm}

Le projet à réaliser en classe préparatoire, ou TIPE, est une épreuve qui parfois fait peur, souvent engendre de la passion, mais qui emblématise une période de recherche et de mise à contribution de nos capacités d'autonomie.
\emph{Quand la logique s'en mêle} emprunte non seulement son nom au projet que j'ai pu réaliser, mais également le sujet et son histoire, afin d'apporter une vision plus large de cette expérience.
Organisé comme un carnet de voyage, il retrace les grandes étapes que j'ai traversé tout en soulignant les périodes interrogatives, la démarche, les rencontres et les discussions vives sur tableau noir pour transmettre une même passion: la recherche. 
L'inspiration, les tournures, et les mécanismes empruntés à \emph{Théorème Vivant} ne sont pas des moindres, alors même que le sujet diffère, parce qu'il s'agit ici de cette même mission.

Passant par des phases de choix de sujet, de confrontation, d'hésitation, de flou, mais aussi de découvertes, d'expériences, de confiance en soi et d'appropriation du contenu scientifique, le texte essaye de retranscrire une version réduite à l'essentiel de mon expérience, toujours en focalisant sur l'avancée du sujet.

\end{document}